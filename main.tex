\documentclass[aspectratio=169,10pt]{beamer}
\usepackage[utf8]{inputenc}
\usepackage[T1,T2A]{fontenc}
\usepackage[serbian]{babel}
\usepackage{cmsrb}
\usetheme{metropolis}
\usepackage{ragged2e}
\apptocmd{\frame}{}{\justifying}{}
\usepackage[export]{adjustbox}
\usepackage{appendixnumberbeamer}
\usepackage{caption}
\usepackage{booktabs,siunitx}
\usepackage{multirow}
\usepackage{forest}
\usepackage{booktabs}
\usepackage[scale=2]{ccicons}
\usepackage{tikz}
\usetikzlibrary{positioning,arrows}
\usepackage{array,tabulary,tabularx,multirow}
\usepackage{graphicx}
\usepackage{subcaption}
\usetikzlibrary{positioning,fadings,through}
\usetikzlibrary{patterns}
\usetikzlibrary{positioning,arrows.meta,bending}
\usepackage{collectbox}
\usepackage{ragged2e}
\usepackage{etoolbox}
\captionsetup{compatibility=false}
\usefonttheme[onlymath]{serif}
\apptocmd{\frame}{}{\justifying}{}
%%%%%%%%%%%%%%%% MOJE %%%%%%%%%%%%%%%%%%
\setbeamercovered{transparent}
\usepackage{tikz}
\usetikzlibrary{overlay-beamer-styles}
 \tikzset{
    highlight on/.style={alt={#1{fill=red!80!black,color=red!80!black}{fill=gray!30!white,color=gray!30!white}}},
}

\makeatletter
\newcommand{\mybox}{%
    \collectbox{%
        \setlength{\fboxsep}{1pt}%
        \fbox{\BOXCONTENT}%
    }%
}
\makeatother

%%%%%%%%%%%%%%%%%%%%%%%%%%%%%%%%%%%%%%%%

\usepackage{booktabs}
\usepackage[scale=2]{ccicons}

\usepackage{pgfplots}
\usepgfplotslibrary{dateplot}
\usepackage{xspace}
\newcommand{\themename}{\textbf{\textsc{metropolis}}\xspace}
\AtBeginDocument{\renewcommand{\figurename}{Слика}}
\AtBeginDocument{\renewcommand{\tablename}{Табела}}

\title{Пример прављења садржаја формата презентације са Beamer класом}
\subtitle{Поднаслов}

\date{{12. октобар 2022.}}
\author{Матеј Томић}
\institute{Истраживачко-развојни центар}

\titlegraphic{\includegraphics[height=1.2cm]{UniBGD.jpeg}\hfill\includegraphics[height=1.3cm]{logo_mf.pdf}}



\begin{document}
\sloppy
  


\maketitle

\begin{frame}{Структура излагања}
  
  \begin{columns}
    \begin{minipage}[t][0.5\textheight]{0.75\textwidth}
    \setbeamertemplate{section in toc}[sections numbered]
      \tableofcontents[sectionstyle=show]
    \end{minipage}\hfill
    \end{columns}
\end{frame}

\section{Напомене}
\begin{frame}{Напомене}

\begin{itemize}

        \item Угледати се на садржај са линка испод \\
        \small{\url{https://www.tug.org/teTeX/tetex-texmfdist/doc/latex/beamer/beameruserguide.pdf}}
\end{itemize}
    
\end{frame}

\section{Цртање помоћу Tikz-а}
\begin{frame}{Блок дијаграми}


\tikzset{
block/.style={
  draw, 
  rectangle, 
  minimum height=1cm, 
  minimum width=3cm, align=center
  }, 
line/.style={->,>=latex'}
}

\begin{figure}[H]
    \centering
    \begin{tikzpicture}[scale=0.7, transform shape]
\node[block] (exp) {\textsf{\normalsize Текст 1}};
\node[block, below =1cm of exp,fill=lightgray] (vdp) {\normalsize \textsf{Вектор брзине} $\vec{U}, \mathbf{U}$};
\node[block, below =1cm of vdp] (DF) {\normalsize \textsf{Закони одржања}\\ \textsf{масе, количине кретања и енергије}};
\node[block, right =2cm of DF] (num) {\normalsize \textsf{Нумерички  прорачун}};
\node[block, below =1cm of num,fill=lightgray] (vdpn) {\normalsize \textsf{Крива зависности}};
\node[block, below =2cm of DF] (compare) {\normalsize \textsf{Поређење и анализа}};

\draw[line] (exp.south) -| (vdp.north);
\draw[line] (vdp.south) -| (DF.north);
\draw[line] (DF.east) |- (num.west);
\draw[line] (num.south) -| (vdpn.north);
\draw[line] (vdpn.south) |- (compare.east);
\draw[line] (vdp.west) -- ++(-1.55,0) |- (compare.west);


\end{tikzpicture}
    \caption{Шематски приказ од интереса.}
    \label{figure6}
\end{figure}


\end{frame}

\begin{frame}{Графикони и дијаграми}

 \begin{figure}
        \centering
       \begin{tikzpicture}
\pgfplotsset{title={Назив графика},
    scale only axis,
    xmin=0, xmax=700000,
    every tick label/.append style={font=\small},
    label style={font=\small}
}

\begin{axis}[legend style={fill=none,nodes={scale=0.6, transform shape},at={(1.3,1)}},/pgf/number format/.cd,
        use comma,
        1000 sep={},
width=5cm,
  height=3cm,
grid=both,
  ymin=0, ymax=2000,
  xlabel={$x\ [\mathrm{m}]$},
   every x tick scale label/.style={at={(rel axis cs:1,0)},anchor=south west,inner sep=1pt},
  ylabel={$y\ [\mathrm{m/s^{2}}]$},
]
\addplot[red, thick]
  coordinates{
(26761,242.774818)
(45270,240.712108)
(67749,195.098826)
(95670,217.26653)
(121056,221.255554)
(157663,221.247728)
(197436,221.663624)
(243490,214.78457)
(294159,218.224656)
(337714,219.381786)
(396945,216.139586)
(458968,219.229738)
(527611,218.525398)
(601308,216.325174)
(662628,218.343164)
  };
  \addlegendentry{$\phi_{1}$}  
\addplot[blue, thick]
  coordinates{
  (26761,657.441576)
(45270,703.01088)
(67749,610.610616)
(95670,647.122608)
(121056,653.628096)
(157663,658.687788)
(197436,644.190624)
(243490,667.241388)
(294159,660.246444)
(337714,663.10002)
(396945,648.601668)
(458968,677.957148)
(527611,675.047736)
(601308,663.265152)
(662628,669.606696)
  };
  \addlegendentry{$\phi_{3}$}  
\addplot[black, thick]
  coordinates{
(26761,1235.6982)
(45270,1136.287548)
(67749,1157.324652)
(95670,1105.726248)
(121056,1163.757672)
(157663,1186.583904)
(197436,1235.08044)
(243490,1222.22628)
(294159,1211.79564)
(337714,1195.87644)
(396945,1147.97628)
(458968,1183.59252)
(527611,1215.2052)
(601308,1210.32252)
(662628,1112.586948)
}; 
\addlegendentry{$\phi_{3}$}  
\end{axis}
\end{tikzpicture}
            \label{fig:mean and std of net34}
    \end{figure}    
\end{frame}


\section{Математички запис}

\begin{frame}{Једначина одржања количине кретања}

Једначина са обележеним бројем
    \begin{equation} \label{eq:1}
    \frac{\partial \rho \vec{U}}{\partial t} + \nabla \cdot (\rho \vec{U} \vec{U}) = \rho \vec{f} - \nabla p + \nabla \cdot \widetilde{T}
    \end{equation}	

Једначина без обележеног броја 
    \begin{equation*}
    \vec{S} = \begin{bmatrix}
    \frac{\partial \vec{p}_{x}}{\partial x} \\[1ex] 
    \frac{\partial \vec{p}_{y}}{\partial y} \\[1ex] 
    \frac{\partial \vec{p}_{z}}{\partial z} \\
    \end{bmatrix},\ \ \widetilde{D} = \begin{bmatrix}
    D_{xx} & \hdots & D_{xz} \\
    \vdots & \ddots & \vdots \\
    D_{zx} & \hdots & D_{zz} \\
    \end{bmatrix},\ \ \widetilde{F} = \begin{bmatrix}
    F_{xx} & \hdots & F_{xz} \\
    \vdots & \ddots & \vdots \\
    F_{zx} & \hdots & F_{zz} \\
    \end{bmatrix}
\end{equation*}
    

\end{frame}

\section{Закључак}

\begin{frame}{За крај}

\begin{itemize}
    \item Цитирати радове на следећи начин \cite{matze}
\end{itemize}


\begin{figure}[H]
\centering
\begin{subfigure}{.5\textwidth}
  \centering
  \includegraphics[width=0.5\linewidth]{example-grid-100x100pt}
  \caption{Показна слика 2}
  \label{fig:Pokazna1}
\end{subfigure}%
\begin{subfigure}{.5\textwidth}
  \centering
  \includegraphics[width=0.5\linewidth]{example-grid-100x100pt}
  \caption{Показна слика 2}
  \label{fig:Pokazna2}
\end{subfigure}
\caption{Показне слике 1 и 2}
\label{fig:PokazneSlike}
\end{figure}   

\begin{itemize}
    \item Позовите се на слику \ref{fig:Pokazna1} и слику \ref{fig:PokazneSlike} 
\end{itemize}

\end{frame}

\begin{frame}[allowframebreaks]{Литература}

  \bibliography{main}
  \bibliographystyle{abbrv}

\end{frame}

\end{document}